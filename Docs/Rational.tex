\documentclass{article}
\usepackage{fancyhdr}
%for headers
\usepackage{textcomp}
%for angles
\usepackage{listings}
%for code listing
\usepackage{xcolor}
%for highlighting
\usepackage{caption}
\usepackage{calc}
%for code boxes
\usepackage{varioref}
%for fancey references
\usepackage{fancyvrb}
%for fancey Verbs

\DeclareCaptionFont{white}{\color{white}}
\DeclareCaptionFormat{listing}{%
	\parbox{\textwidth}{\colorbox{gray}{\parbox{\textwidth}{#1#2#3}}\vskip-4pt}}
\newlength\tdima \newlength\tdimb \setlength\tdima{ \fboxsep+\fboxrule} \setlength\tdimb{-\fboxsep+\fboxrule}

\renewcommand\lstlistingname{Example}


\newcommand{\theobj}{\protect\Verb+RationalImpl<T> +}
\newcommand{\thedocname}{The \theobj Class}
\rhead{\hfill {\thedocname}}
\pagestyle{fancy}
\cfoot{\thedocname :\quad Page \thepage}

\begin{document}

\tableofcontents



\definecolor{codegreen}{rgb}{0,0.6,0}
\definecolor{codegray}{rgb}{0.5,0.5,0.5}
\definecolor{codepurple}{rgb}{0.58,0,0.82}
\definecolor{codetext}{rgb}{0.7,0.7,0.7}
\definecolor{backcolour}{rgb}{0.8,0.8,0.8}

\lstdefinestyle{cppstyle}{
	frame=tlrb,
	xleftmargin=\tdima,
	xrightmargin=\tdimb,
	backgroundcolor=\color{backcolour},
	commentstyle=\color{codegreen}\ttfamily,
	keywordstyle=\color{blue}\ttfamily,
	numberstyle=\scriptsize\color{codegray},
	stringstyle=\color{red}\ttfamily,
	basicstyle=\footnotesize\ttfamily,
	breakatwhitespace=false,         
	breaklines=true,                  
	keepspaces=true,                 
	numbers=left,                    
	numbersep=5pt,                  
	showspaces=false,                
	showstringspaces=false,
	showtabs=false,                  
	tabsize=2
}
\captionsetup[lstlisting]{format=listing,labelfont=white,textfont=white}
\lstset{style=cppstyle}

\section{The \theobj Class}

The \theobj Class is a rational (fractions) class. 

\subsection{Using \theobj}

Its very simple to use.  \Verb+Rational(1,2)+ will be a rational with data as the type \Verb+std::ptrdiff_t+. Using rational with a data type that cannot be used with the unary \Verb+-+ operator is undefined.

\begin{lstlisting}[language=C++, label=code:1, caption=Creating organized rationals with \theobj]
cg::RationalImpl<int16_t> r(4, 8); //numerator and denominator are type int16_t.
cg::Rational r2(4, 8);//numerator and denominator are type std::ptrdiff_t.  
r2.AutoSimp(true); //Will also simplify automatically.
\end{lstlisting}

Creating the \theobj and using the \Verb+r.AutoSimp(true)+ will cause it to attempt to simplify the object after each operation.  It is recomended to use typedef or using declarations for this class (such as \Verb+using BigRational = RationalImpl<SomeBigNumClass>;+).

\paragraph{Why?} \hfill \\
The reason for this classes existance is that some developers may want floating point number that is 100\% accurate and can be used without having to deal with the rounding of floats and double. Combined with a int64\underline{ }t data type, The rational object can be a fractional value with 100\% accuracy as far as the datatype can handle. \textbf{Bigger user defined datatypes may also be used as long as they have the operators for math defined.}

\subsection{From double to \theobj}

Creating a \theobj from a double value is easy.

\begin{lstlisting}[language=C++, label=code:1, caption=Creating organized rationals with \theobj]
auto r = cg::Rational::Make(0.444329, 9); //9 is the amount of decimals to keep. r == (444329000 / 1000000000)
r.AutoSimp(true); // autosimp is now on: r == (444329 / 1000000)
\end{lstlisting}

The simplification of the rational uses the euclidean GCD algorithm, and the LCM algorithm uses the GCD to speed things up, so both use euclidean algorithms.

\subsection{Inverses of \theobj}

The \theobj class has 4 functions dealing with inverses. \Verb+AInverse(), Opposite()+ and \Verb+MInverse(), Reciprocal()+.


\begin{lstlisting}[language=C++, label=code:1, caption=Creating organized rationals with \theobj]
cg::Rational r2(4, 8);
auto X = r2.Reciprocal(); //X now = rational of (8/4).
r2.MakeReciprocal();      //r2 now is (8/4).
auto Y = r2.Opposite();   //Y is now -(8/4).
r2.MakeOpposite();        //r2 is now -(8/4).
r2.Simplify();            //r2 is now -(2/1).
\end{lstlisting}

For the rational number, \Verb+MInverse()+ is the same as \Verb+Reciprocal()+ and \Verb+AInverse()+ is the same as \Verb+Opposite()+. Each function has a version that is \Verb+r.Make***+ where the stars are Opposite, Reciprocal, AInverse, MInverse.

\subsection{Scaling Up and Down}

The rational can also be scaled, which is multiplying or dividing the numerator and denominator by the same value.


\begin{lstlisting}[language=C++, label=code:1, caption=Creating organized rationals with \theobj]
cg::Rational r2(4, 8);
r2.ScaleUp(3);         //r2 = (12, 24)
r2.ScaleDown(3);       //r2 = (4, 8) .... returns true
r2.ScaleDown(3);       //return false, nothing done, 3 is not evenly divisable by the numerator and denominator.
\end{lstlisting}
If the values are not divisible by the \Verb+ScaleDown()+ argument, then nothing happens and \Verb+false+ is returned. If they are divisible (both of them), they are scaled down and \Verb+true+ is returned.

\subsection{Math with \theobj}

All math and comparisons are overloaded.

\begin{lstlisting}[language=C++, label=code:1, caption=Creating organized rationals with \theobj]
cg::Rational r2(4, 8);
cg::Rational r3(4, 9);
auto X = r2 (+,-,*,/) r3; //Works as expected
auto Y = r2 (<,<=, ==, !=, >, >=) r3 //Works as expected.
r2 (+=, -=, *=, /=) r3;    //Works as expected.
r2 (+=, -=, *=, /=) 97;    //Works as expected.
++r2; r2++; --r2; r2--; -r2; //All works as expected.
\end{lstlisting}

All operators except the modulo operator will work properly.

\subsection{Evaluation}

The \theobj class can be evaluated to another number, provided that the data type can take a constructor of type T, and T has the /= operator working properly.


\begin{lstlisting}[language=C++, label=code:1, caption=Creating organized rationals with \theobj]
cg::Rational r2(4, 8);
auto someBigDecimal = r2.Eval<SomeBigDecimalType>();
\end{lstlisting}

The accuracy of the Eval functions is only as good as the /= operator for the type given to it.

\end{document}

























