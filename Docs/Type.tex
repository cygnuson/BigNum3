\documentclass{article}
\usepackage{fancyhdr}
%for headers
\usepackage{textcomp}
%for angles
\usepackage{listings}
%for code listing
\usepackage{xcolor}
%for highlighting
\usepackage{caption}
\usepackage{calc}
%for code boxes
\usepackage{varioref}
%for fancey references
\usepackage{fancyvrb}
%for fancey Verbs

\DeclareCaptionFont{white}{\color{white}}
\DeclareCaptionFormat{listing}{%
	\parbox{\textwidth}{\colorbox{gray}{\parbox{\textwidth}{#1#2#3}}\vskip-4pt}}
\newlength\tdima \newlength\tdimb \setlength\tdima{ \fboxsep+\fboxrule} \setlength\tdimb{-\fboxsep+\fboxrule}

\renewcommand\lstlistingname{Example}


\newcommand{\theobj}{\protect\Verb+DemoteType +}
\newcommand{\theobjt}{\protect\Verb+PromoteType +}
\newcommand{\thedocname}{The \theobj and \theobjt Class}
\rhead{\hfill {\thedocname}}
\pagestyle{fancy}
\cfoot{\thedocname :\quad Page \thepage}

\begin{document}

\tableofcontents



\definecolor{codegreen}{rgb}{0,0.6,0}
\definecolor{codegray}{rgb}{0.5,0.5,0.5}
\definecolor{codepurple}{rgb}{0.58,0,0.82}
\definecolor{codetext}{rgb}{0.7,0.7,0.7}
\definecolor{backcolour}{rgb}{0.8,0.8,0.8}

\lstdefinestyle{cppstyle}{
	frame=tlrb,
	xleftmargin=\tdima,
	xrightmargin=\tdimb,
	backgroundcolor=\color{backcolour},
	commentstyle=\color{codegreen}\ttfamily,
	keywordstyle=\color{blue}\ttfamily,
	numberstyle=\scriptsize\color{codegray},
	stringstyle=\color{red}\ttfamily,
	basicstyle=\footnotesize\ttfamily,
	breakatwhitespace=false,         
	breaklines=true,                  
	keepspaces=true,                 
	numbers=left,                    
	numbersep=5pt,                  
	showspaces=false,                
	showstringspaces=false,
	showtabs=false,                  
	tabsize=2
}
\captionsetup[lstlisting]{format=listing,labelfont=white,textfont=white}
\lstset{style=cppstyle}

\section{The \theobj and \theobjt Class}

The \theobj \theobjt classes are for transforming types into other types relative to their size. See the next subsections for examples.

\subsection{Using \theobj}

For any type \Verb+T+ where \Verb+sizeof(T) > 1+,  \Verb+cg::DemoteType<T>::Type+ is a primitive type that is exactly half the size of T. See Example \vref{code:1}.

\begin{lstlisting}[language=C++, label=code:1, caption=Demoting types]
using HalfType = typename cg::DemoteType<uint32_t>::Type;
/**Get reference access to the least significant uint16_t digit.
\param n The number to access.
\return A reference as uint16_t that is the less significant part of the number.*/
HalfType& GetLoPart(uint32_t& n)
{
	if(cg::Endian::little)
		return *((uint16_t) &n);
	else
		return *(()(uint16_t) &n) + 1);
}
\end{lstlisting}

When \Verb+T+ is a type such that \Verb+sizeof(T) == 1+, the type \Verb+cg::DemoteType<T>::Type+ does not exist and will cause a compiler error.

\subsection{Using \theobjt}

For any type T where \Verb+sizeof(T) < 8+ \Verb+typename cg::PromoteType<T>::Type+ is a primitive type that is exactly double the size of T. See Example \vref{code:2}.

\begin{lstlisting}[language=C++, label=code:2, caption=Promoting Types]
using DoubleType = typename cg::PromoteType<uint32_t>::Type;
auto oSize = sizeof(uint32_t);   // oSize = 4
auto nSize = sizeof(DoubleType); //nSize = 8, DoubleType = uint64_t
\end{lstlisting}


When \Verb+T+ is a type such that \Verb+sizeof(T) == 8+, the type \Verb+cg::PromoteType<T>::Type+ does not exist and will cause a compiler error.

\subsection{Final Thoughts}

A great example is in the source code for the \Verb+Num<T>+ class header. The member functions use the endian class and \theobj class to detect system endianess to properly decompose larger data types into multiple smaller data types as a reference or copy.


\end{document}

























