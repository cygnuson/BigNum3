\documentclass{article}
\usepackage{fancyhdr}
%for headers
\usepackage{textcomp}
%for angles
\usepackage{listings}
%for code listing
\usepackage{xcolor}
%for highlighting
\usepackage{caption}
\usepackage{calc}
%for code boxes
\usepackage{varioref}
%for fancey references
\usepackage{fancyvrb}
%for fancey Verbs

\DeclareCaptionFont{white}{\color{white}}
\DeclareCaptionFormat{listing}{%
	\parbox{\textwidth}{\colorbox{gray}{\parbox{\textwidth}{#1#2#3}}\vskip-4pt}}
\newlength\tdima \newlength\tdimb \setlength\tdima{ \fboxsep+\fboxrule} \setlength\tdimb{-\fboxsep+\fboxrule}

\renewcommand\lstlistingname{Example}


\newcommand{\theobj}{\protect\Verb+Endian +}
\newcommand{\thedocname}{The \theobj Class}
\rhead{\hfill {\thedocname}}
\pagestyle{fancy}
\cfoot{\thedocname :\quad Page \thepage}

\begin{document}

\tableofcontents



\definecolor{codegreen}{rgb}{0,0.6,0}
\definecolor{codegray}{rgb}{0.5,0.5,0.5}
\definecolor{codepurple}{rgb}{0.58,0,0.82}
\definecolor{codetext}{rgb}{0.7,0.7,0.7}
\definecolor{backcolour}{rgb}{0.8,0.8,0.8}

\lstdefinestyle{cppstyle}{
	frame=tlrb,
	xleftmargin=\tdima,
	xrightmargin=\tdimb,
	backgroundcolor=\color{backcolour},
	commentstyle=\color{codegreen}\ttfamily,
	keywordstyle=\color{blue}\ttfamily,
	numberstyle=\scriptsize\color{codegray},
	stringstyle=\color{red}\ttfamily,
	basicstyle=\footnotesize\ttfamily,
	breakatwhitespace=false,         
	breaklines=true,                  
	keepspaces=true,                 
	numbers=left,                    
	numbersep=5pt,                  
	showspaces=false,                
	showstringspaces=false,
	showtabs=false,                  
	tabsize=2
}
\captionsetup[lstlisting]{format=listing,labelfont=white,textfont=white}
\lstset{style=cppstyle}

\section{The \theobj Class}

The \theobj Class is a simple little class for testing endianess on a system. Its used primarily (for this project) in determining the placement of HI and LO digits of number data.  For example an \Verb+uint32_t+ has a size of four bytes, and the first two and latter two bytes are each a digit of type \Verb+uint16_t+.  Which is the Hi or Lo digit? Use the \theobj class to find out.

\subsection{Using \theobj}

To use \theobj just make use of its members directly as seen in Example \vref{code:test}.

\begin{lstlisting}[language=C++, label=code:test, caption=Testing Endianess]
uint32_t num = 1048833;  
// 1048833 == [0000 0000 0001 0000][0000 0001 0000 0001] 
uint16_t* numLoPart = nullptr; 
if(cg::Endian::litte)
	/*system is little endian*/
	numLoPart = (uint16_t) &num;
else
	/*system is big endian*/
	numLoPart = ((uint16_t) &num) + 1;  //Move to the first uint16_t data plus 1 unit of uint16_t in the significance direction.
*numLoPart = 0; // num == 1048576
// 1048576 == [0000 0000 0001 0000][0000 0000 0000 0000] 
\end{lstlisting}

\subsection{Final Thoughts}

A great example is in the source code for the \Verb+Num<T>+ class header. The member functions use the endian class to detect system endianess to properly decompose larger data types into multiple smaller data types as a reference or copy.


\end{document}

























